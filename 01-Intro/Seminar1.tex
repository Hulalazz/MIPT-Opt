\documentclass[12pt]{beamer}
\usepackage{../latex-sty/mypres}
\usepackage[utf8]{inputenc}
\usepackage[russian]{babel}
\usepackage[T2A]{fontenc}

\expandafter\def\expandafter\insertshorttitle\expandafter{%
  \insertshorttitle\hfill%
  \insertframenumber\,/\,\inserttotalframenumber}
  
\title[Семинар 1]{Методы оптимизации. \\
Семинар 1. Введение.}
\author{Александр Катруца}
\institute{Московский физико-технический институт,\\
Факультет Управления и Прикладной Математики} 
\date{\today}

\begin{document}
\begin{frame}
\titlepage
\end{frame}

\begin{frame}{Вопросы к студентам}
\begin{itemize}
\item Имя
\item Кафедра
\item Знание \TeX/\LaTeX
\item Ожидания от курса
\end{itemize}

\end{frame}

\begin{frame}{Зачем этот курс?}
\begin{itemize}
\item Формализация задачи выбора элемента из множества
\item Обоснование правильности принятия решения
\item Разнообразные приложения:
\begin{itemize}
\item машинное обучение: классификация, кластеризация, регрессия
\item молекулярное моделирование
\item анализ рисков
\item выбор активов (portfolio optimization)
\item оптимальное управление
\item обработка сигналов
\item оценка параметров в статистике 
\item и другие\footnote{\url{http://www.cvxpy.org/en/latest/examples/index.html}}
\end{itemize}
\end{itemize}
\end{frame}

\begin{frame}{О чём этот курс?}
Первый семестр (теоретический):
\begin{itemize}
\item Основы выпуклого анализа
\item Теория двойственности
\item Условия оптимальности
\end{itemize}

Второй семестр (практический):
\begin{itemize}
\item Методы безусловной минимизации первого и второго порядка 
\item Методы условной оптимизации
\item Линейное программирование: симплекс-метод и пр.
\item Оптимальные методы
\item ...
\end{itemize}
\end{frame}

\begin{frame}{План на семестр}
\begin{itemize}
\item Семинар и лекция раз в неделю
\item 2 задания
\item Итоговая контрольная в конце семестра \\ (и промежуточная в середине семестра)
\item Экзамен в конце семестра. Среднее арифметическое от:
\begin{itemize}
\item оценки за работу в семестре
\item ответа на экзамене
\end{itemize}
\item Миниконтрольные в начале каждого семинара
\item Домашнее задание на каждый семинар~--- \LaTeX
\end{itemize}
\end{frame}

\begin{frame}{Предварительные навыки}
\begin{itemize}
\item Линейная алгебра
\item Математический анализ
\item Программирование: Python (NumPy, SciPy, CVXPY) или MATLAB
\item Элементы вычислительной математики
\end{itemize}
\end{frame}

\begin{frame}{Методология}
Основные этапы использования методов оптимизации при решении реальных задач:
\begin{enumerate}
\item Определение целевой функции
\item Определение допустимого множества решений
\item Постановка и анализ оптимизационной задачи
\item Выбор наилучшего алгоритма для решения поставленной задачи
\item Реализация алгоритма и проверка его корректности
\end{enumerate}

\end{frame}

\begin{frame}{Постановка задачи}
\begin{equation*}
\begin{split}
&\min\limits_{\bx \in X} f_0(\bx)\\
\text{s.t. } & f_i(\bx) = 0, \; i = 1,\ldots,p\\
& f_j(\bx) \leq 0, \; j = p+1,\ldots,m,
\end{split}
\end{equation*}
\begin{itemize}
\item $\bx \in \bbR^n$~--- искомый вектор
\item $f_0(\bx): \bbR^n \rightarrow \bbR$~--- целевая функция
\item $f_k(\bx): \bbR^n \rightarrow \bbR$~--- функции ограничений
\end{itemize}
Пример: выбор объектов для вложения денег и определение в какой объект сколько вкладывать
\begin{itemize}
\item $\bx$~--- размер инвестиций в каждый актив
\item $f_0$~--- суммарный риск или вариация прибыли
\item $f_k$~--- бюджетные ограничения, min/max вложения в актив, минимально допустимая прибыль
\end{itemize}

\end{frame}

\begin{frame}{Как решать?}
В общем случае:
\begin{itemize}
\item NP-полные
\item {\small рандомизированные алгоритмы: время vs стабильность}
\end{itemize}

{\small НО определённые классы задач могут быть решены быстро!}

\begin{itemize}
\item Линейное программирование
\item Метод наименьших квадратов
\item Малоранговое приближение порядка $k$
\item Выпуклая оптимизация
\end{itemize}
\end{frame}

\begin{frame}{Линейное программирование}
\begin{equation}
\begin{split}
&\min\limits_{\bx \in \bbR^n} \bc^{\T}\bx\\
\text{s.t. } & \ba^{\T}_i \bx \leq c_i, \; i = 1,\ldots, m
\end{split}
\label{eq::lin_prog}
\end{equation}
\begin{itemize}
\item нет аналитического решения
\item существуют эффективные алгоритмы
\item разработанная технология
\item симплекс-метод для решения задачи~\eqref{eq::lin_prog} входит в Top-10 алгоритмов XX века\footnote{\url{https://www.siam.org/pdf/news/637.pdf}}
\end{itemize}
\end{frame}

\begin{frame}{Метод наименьших квадратов}
\begin{equation}
\min\limits_{\bx \in \bbR^n} \|\bA\bx - \mathbf{b} \|^2_2,
\label{eq::least_sq}
\end{equation}
где $\bA \in \bbR^{m \times n}$ и $\mathbf{b} \in \bbR^m$.
\begin{itemize}
\item имеет аналитическое решение: $\bx^* = (\bA^{\T}\bA)^{-1}\bA^{\T}\mathbf{b}$
\item существуют эффективные алгоритмы
\item разработанная технология
\item имеет статистическую интерпретацию
\end{itemize}
\end{frame}

\begin{frame}{Малоранговое приближение ранга $k$}

\begin{equation}
\begin{split}
&\min\limits_{\bX \in \bbR^{m \times n}} \|\bA - \bX \|_F \\
\text{s.t. } & \text{rank}(\bX) \leq k
\end{split}
\label{eq::lowrank}
\end{equation}

\begin{Theorem}[Eckart–Young, 1993]
Пусть $\bA = \bU\bSigma\bV^{\T}$~--- сингулярное разложение матрицы $\bA$, где $\bU = [\bU_k, \bU_{r-k}] \in \bbR^{m \times r}$, $\bSigma = \diag{\sigma_1, \ldots, \sigma_k, \ldots, \sigma_r}$, $\bV = [\bV_k, \bV_{r - k}] \in \bbR^{n \times r}$ и $r = \text{rank}(\bA)$.  Тогда решение задачи~\eqref{eq::lowrank} можно записать в виде:
\[
\bX = \hat{\bU} \hat{\bSigma} \hat{\bV}^{\T},
\]
где $\hat{\bU} \in \bbR^{m \times k}$, $\hat{\bSigma} = \diag{\sigma_1, \ldots, \sigma_k}$, $\hat{\bV} \in \bbR^{n \times k}$.
\end{Theorem}
\small Алгоритм вычисления сингулярного разложения и быстрый, и устойчивый.
\end{frame}

\begin{frame}{Выпуклая оптимизация}
\begin{equation}
\begin{split}
&\min\limits_{\bx \in \bbR^n} f_0(\bx)\\
\text{s.t. } & f_i(\bx) \leq b_i, \; i = 1,\ldots, m
\end{split}
\label{eq::conv_opt}
\end{equation}
\begin{itemize}
\item $f_0, f_i$~--- выпуклые функции:
\[
f(\alpha \bx_1 + \beta \bx_2) \leq \alpha f(\bx_1) + \beta f(\bx_2),
\]
где $\alpha, \beta \geq 0$ и $\alpha + \beta = 1$.
\item нет аналитического решения
\item существуют эффективные алгоритмы
\item часто сложно <<увидеть>> задачу выпуклой оптимизации
\item существуют приёмы для преобразования задачи к виду~\eqref{eq::conv_opt}
\end{itemize}
\end{frame}

\begin{frame}{Какая задача проще?}

\begin{columns}[T]
\begin{column}{0.5\textwidth}
Поиск независимого под-алфавит максимальной мощности
\begin{equation*}
\begin{split}
&\max\limits_{\bx \in \bbR^n} \sum\limits_{i=1}^n x_i\\
\text{s.t. } & x_i^2 - x_i = 0 \quad i = 1,\ldots, n\\
& x_ix_j = 0 \qquad \forall (i, j) \in \Gamma,
\end{split}
\end{equation*}
где $\Gamma$~--- множество пар
\end{column}
~
\begin{column}{0.5\textwidth}
\small
Truss design
\begin{equation*}
\begin{split}
& \min -2\sum\limits_{i=1}^k\sum\limits_{j=1}^m c_{ij} x_{ij} + x_{00}\\
\text{s.t. } & \sum\limits_{i=1}^k x_i = 1\\
& \lambda_{\min}(\bA) \geq 0,
\end{split}
\end{equation*}
\scriptsize
где $\bA = 
\begin{bmatrix}
x_1 & \ldots & \ldots & \sum\limits_{j=1}^m b_{pj}x_{1j}\\
\vdots & \ddots & & \vdots\\
 & & x_k & \sum\limits_{j=1}^m b_{pj}x_{kj}\\
\sum\limits_{j=1}^m b_{pj}x_{1j} & \ldots & \sum\limits_{j=1}^m b_{pj}x_{kj} & x_{00}
\end{bmatrix}
$
\end{column}
\end{columns}

\end{frame}

\begin{frame}{Почему выпуклость так важна?}
\begin{block}{R. Tyrrell Rockafellar (1935~---)}
The great watershed in optimization is not between linearity and
non-linearity, but convexity and non-convexity.
\end{block}
\begin{itemize}
\item Локальный оптимум является глобальным
\item Необходимое условие оптимальности является достаточным
\end{itemize}
Вопросы:
\begin{itemize}
\item Любую ли задачу выпуклой оптимизации можно эффективно решить?
\item Можно ли эффективно решить невыпуклые задачи оптимизации?
\end{itemize}
\end{frame}

\begin{frame}{Резюме}
\begin{itemize}
\item Организация работы
\item Предмет курса по оптимизации
\item Общая формулировка оптимизационной задачи
\item Классические оптимизационные задачи
\end{itemize}
\end{frame}

\end{document}
