\documentclass[12pt,russian]{beamer}
\usepackage{../latex-sty/mypres}
\usepackage[utf8]{inputenc}
\usepackage[english]{babel}

\expandafter\def\expandafter\insertshorttitle\expandafter{%
  \insertshorttitle\hfill%
  \insertframenumber\,/\,\inserttotalframenumber}
\title[Seminar 2]{Optimization Methods. \\
Seminar 2. Convex sets.}
\author{Alexandr Katrutsa}
\institute{Moscow Institute of Physics and Technology,\\
Department of Computational and Applied Mathematics} 
\date{September 12, 2016}
\begin{document}
\begin{frame}
\maketitle
\end{frame}

\begin{frame}{Reminder}

\begin{itemize}
\item Objective of the Optimization Methods course
\item General optimization problem statement
\item Examples of optimization problems: 
\begin{itemize}
\item linear programming
\item least squares problem
\item convex optimization problems
\end{itemize}
\item Why convex optimization problems are good?
\end{itemize}
\end{frame}

\begin{frame}{Affine sets}
\small
\begin{block}{Affine set}
Let $A$ be affine set if for any $x_1$, $x_2 \in A$ and $\theta \in \bbR$ point $\theta x_1 + (1 - \theta)x_2 \in A$.
\end{block}
Examples: $\bbR^n$, hyperplane, single point.

\begin{block}{Affine combination of points}
Assume $x_1, \ldots, x_k \in G$, then point $\theta_1 x_1 + \ldots + \theta_k x_k$ is called affine combination of the points $x_1,\ldots,x_k$ if {\color{red}{$\sum\limits_{i=1}^k \theta_i = 1$}}.
\end{block}

\begin{block}{Affine hull of set}
A set $\left\{ \sum\limits_{i=1}^k \theta_i x_i \; | \; x_i \in G, {\color{red}{\sum\limits_{i=1}^k \theta_i = 1}} \right\}$ is called affine hull of set $G$ and is denoted by \textbf{aff}(G).
\end{block}
\end{frame}

\begin{frame}{Claims}

\begin{block}{Claim 1}
A set $G$ is affine if and only if it contains all affine combinations of points from $G$.
\end{block}

\begin{block}{Claim 2}
A set $G$ is affine if and only if it can be represented in the form  $G = \{\bx | \bA\bx = \mathbf{b} \}$.
\end{block}
\end{frame}

\begin{frame}{Convex set}
\small
\begin{block}{Convex set}
Let $C$ be a convex set if 
\vspace{-4mm}
\[
\forall x_1, \; x_2 \in C, \theta \in [0, 1] \rightarrow \theta x_1 + (1 - \theta)x_2 \in C.
\vspace{-4mm}
\]
$\emptyset$ и $\{ x_0 \}$ are also convex by definition.
\end{block}
Examples: $\bbR^n$, affine set, half-open segment, segment.

\begin{block}{Convex combination of points}
Assume $x_1, \ldots, x_k \in G$, then point $\theta_1 x_1 + \ldots + \theta_k x_k$ is called convex combination of points $x_1,\ldots,x_k$ if {\color{red}{$\sum\limits_{i=1}^k \theta_i = 1, \;\theta_i \geq 0$}}.
\end{block}

\begin{block}{Convex hull}
A set $\left\{ \sum\limits_{i=1}^k \theta_i x_i \; | \; x_i \in G, {\color{red}{\sum\limits_{i=1}^k \theta_i = 1, \theta_i \geq 0}} \right\}$ is called convex combination of a set $G$ and is denoted by \textbf{conv}(G).
\end{block}

\end{frame}

\begin{frame}{Operations that preserve convexity}
\begin{itemize}
\item Intersection of any number (finite or infinite) of convex sets is convex set.
\item Image of convex set under affine function is convex function
\item Linear combination of convex sets is convex
\item Direct product of convex sets is convex
\end{itemize}
\end{frame}

\begin{frame}{Examples}
Check if the following sets are affine and/or convex:
\begin{itemize}
\item Half-space: $\{ \bx | \ba^{\T} \bx \leq c \}$
\item Polyhedron: $\{ \bx | \bA\bx \preceq \mathbf{b}, \; \bC\bx = 0 \}$
\item A ball induced by norm in $\bbR^n$: $B(r, x_c) = \{ x \; | \; \| x - x_c \| \leq r \}$
\item Ellipsoid: $\mathcal{E}(x_c, \bP, r) = \{ x \; | \; (x - x_c)^{\T}\bP^{-1} (x - x_c) \leq r \}$
\item A set of symmetric and positive-definite matrices: $\bS^n_+ = \{ \bX \in \bbR^{n \times n} \; | \; \bX^{\T} = \bX, \; \bX \succeq 0 \}$
\item $\{ \bX \in \bbR^{n \times n} \; | \; \Tr(\bX) = const \}$
\item Hyberbolic set: $\{ \bx \in \bbR^n_+ \; | \; \prod\limits_{i=1}^n x_i \geq 1 \}$
\end{itemize}
\end{frame}

\begin{frame}{Cone}
\small
\begin{block}{Cone (convex)}
Let a set $C$ be a cone (convex cone), if 
\vspace{-4mm}
\begin{equation*}
\begin{split}
& \forall x \in C, \theta \geq 0 \rightarrow \theta x \in C \\
& (\forall x_1, x_2 \in C, \theta_1, \theta_2 \geq 0 \rightarrow \theta_1 x_1 + \theta_2 x_2 \in C)
\end{split}
\end{equation*}
\vspace{-4mm}
\end{block}
Examples: $\bbR^n$, affine set with 0, half-open segment.
\begin{block}{Conical (non-negative) combination of points}
Assume $x_1, \ldots, x_k \in G$, then point $\theta_1 x_1 + \ldots + \theta_k x_k$ is called conic (non-negative) combination of points $x_1,\ldots,x_k$ if {\color{red}{$\theta_i \geq 0$}}.
\end{block}

\begin{block}{Conical hull}
A set $\left\{ \sum\limits_{i=1}^k \theta_i x_i \; | \; x_i \in G, {\color{red}{\theta_i \geq 0}} \right\}$ is called conical hull of a set $G$ and is denoted by \textbf{cone}(G).
\end{block}
\end{frame}

\begin{frame}{Examples}
\begin{itemize}
\item $\bS^n_+$
\item Normal cone: $\{ (\bx, t) \in \bbR^{n+1} \; | \; \| \bx \| \leq t \}$ 

In case of $\ell_2$ norm it is called second-order cone or Lorentz cone
\item Some special cases
\end{itemize}
\end{frame}

\begin{frame}{Recap}
\begin{itemize}
\item Affine set
\item Convex set
\item Cone
\item Methods to check properties of given set
\end{itemize}
\end{frame}

\end{document}
