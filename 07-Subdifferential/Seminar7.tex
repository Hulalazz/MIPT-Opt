\documentclass[12pt]{beamer}
\usepackage{../latex-sty/mypres}
\usepackage[utf8]{inputenc}
\usepackage[T2A]{fontenc}
\usepackage[russian]{babel}

\expandafter\def\expandafter\insertshorttitle\expandafter{%
  \insertshorttitle\hfill%
  \insertframenumber\,/\,\inserttotalframenumber}
\title[Семинар 7]{Методы оптимизации. \\
 Семинар 7. Субдифференциал.}
\author{Александр Катруца}
\institute{Московский физико-технический институт,\\
Факультет Управления и Прикладной Математики} 
\date{\today}

\begin{document}
\begin{frame}
\maketitle
\end{frame}

\begin{frame}{Напоминание}
\begin{itemize}
\item Выпуклая функция
\item Надграфик и множество подуровня функции
\item Критерии выпуклости функции
\item Неравенство Йенсена
\end{itemize}
\end{frame}

\begin{frame}{Мотивация}
\begin{block}{Зачем?}
Важным свойством непрерывной выпуклой функции $f$ является то, что в выбранной точке $\bx$ для всех $\by \in \text{dom } f$ выполнено неравенство:
\vspace{-3mm} 
\[
f(\by) - f(\bx) \geq \langle \ba, \by - \bx \rangle
\vspace{-4mm}
\]
для некоторого вектора $\ba$, то есть касательная к графику функции является {\color{red}{глобальной}} оценкой снизу для функции. 
\end{block}

\begin{itemize}
\item Если $f$ дифференцируема, то $\ba = \nabla f(\by)$.
\item Что делать, если $f$ недифференцируема?
\end{itemize}

\end{frame}

\begin{frame}{Определение}
\begin{block}{Субградиент}
Вектор $\ba$ называется субградиентом функции $f: X \rightarrow \bbR^n$ в точке $\bx$, если 
\vspace{-3mm}
\[
f(\by) - f(\bx) \geq \langle \ba, \by - \bx \rangle
\]
для всех $\by \in X$.
\end{block}

\begin{block}{Субдифференциал}
Множество субградиентов функции $f$ в точке $\bx$ называется субдифференциалом $f$ в $\bx$ и обозначается $\partial f(\bx)$.
\end{block}
\end{frame}

\begin{frame}{Полезные факты}
\begin{block}{Теорема Моро-Рокафеллара}
Пусть $f_i(\bx)$~--- выпуклые функции на выпуклых множествах $G_i, \; i = 1,\ldots,n$. 
Тогда, если $\bigcap\limits_{i=1}^n \text{relint} (G_i) \neq \varnothing$ то функция $f(\bx) = \sum\limits_{i=1}^n a_i f_i(\bx), \; a_i > 0$ имеет субдифференциал $\partial_G f(\bx)$ на множестве $G = \bigcap\limits_{i=1}^n G_i$ и $\partial_G f(\bx) = \sum\limits_{i=1}^n a_i \partial_{G_i} f_i(\bx)$. 
\end{block}

\begin{block}{Если функция~--- максимум}
Если $f(\bx) = \max\limits_{i=1,\ldots,m}(f_i(\bx))$, где $f_i(\bx)$ выпуклы, тогда 
\pause
$\partial_G f(\bx) = \text{Conv} \left(\bigcup\limits_{i \in \calJ(\bx)} \partial_G f_i(\bx)\right)$, где $\calJ(\bx) = \{ i = 1,\ldots, m | f_i(\bx) = f(\bx) \}$
\end{block}
\end{frame}

\begin{frame}{Примеры}
Найдите субдифференциал для следующих функций.
\begin{itemize}
\item Модуль: $f(x) = |x|$
\item Норма $\ell_2$: $f(\bx) = \| \bx \|_2$
\item Скалярный максимум: $f(x) = \max(e^x, 1 - x)$
\item Векторный максимум: $f(\bx) = |\bc^{\T}\bx|$
\item $f(\bx) = |\bc^{\T}_1\bx| + |\bc^{\T}_2\bx|$
\end{itemize}
\end{frame}

\begin{frame}{Условный субдифференциал}
\footnotesize
\begin{block}{Определение}
\footnotesize
Множество 
$
\{ \ba |  f(\bx) - f(\bx_0) \geq \langle \ba, \bx - \bx_0 \rangle, \; \forall \bx \in X \}
$ 
называется субдифференциалом $f$ в $\bx_0$ на множестве $X$ и обозначается $\partial_X f(\bx_0)$.
\end{block}
%Если функция $f$ определена на множестве $X \subset \bbR^n$, то как определить субдифференциал в граничных точках?
\begin{block}{Как перейти от безусловного субдифференциала к условному?}
\footnotesize
Если $f$~--- выпуклая функция, то рассмотрим функцию $g(\bx) = f(\bx) + \delta(\bx | X)$, которая тоже выпуклая.
Тогда
\vspace{-2mm} 
\[
\partial g (\bx_0) = \partial_X f(\bx_0) = \partial f(\bx_0) + \partial \delta(\bx_0 | X).
\vspace{-3mm} 
\]
Найдём $\partial \delta(\bx_0 | X)$:
\vspace{-3mm}
\[
\delta(\bx |X) - \delta(\bx_0 |X) \mathop{=}\limits^{\bx \in X} 0 \geq \langle \ba, \bx - \bx_0 \rangle
%\vspace{-3mm}
\]
\end{block}

\begin{block}{Нормальный конус}
\footnotesize
Множество $N(\bx_0 | X) = \{ \ba | \langle \ba, \bx - \bx_0 \rangle \leq 0, \; \forall \bx \in X \}$ называется нормальным конусом к множеству $X$ в точке $\bx_0$.
\end{block}
Тогда $\partial_X f(\bx_0) = \partial f(\bx_0) + N(\bx_0 | X)$  
\end{frame}

\begin{frame}{Примеры}
\begin{itemize}
\item $f(x) = |x|$, $X = \{-1 \leq x \leq 1 \}$
\item $f(\bx) = |x_1 - x_2|$, $X = \{ \bx | \| \bx \|^2_2 \leq 2 \}$
%\item $f(\bx) = |x_1 - x_2| + |x_1 + x_2|$, $X = \{ \bx | \| \bx \|^2_2 \leq 2 \}$
\end{itemize}
\end{frame}

\begin{frame}{Резюме}
\begin{itemize}
\item Субградиент
\item Субдифференциал
\item Условный субдифференциал
\item Методы вычислений
\end{itemize}
\end{frame}

\end{document}
