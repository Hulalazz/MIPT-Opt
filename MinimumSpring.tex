\documentclass[12pt]{article}
\usepackage[utf8]{inputenc}
\usepackage[russian]{babel}
\usepackage[T2A]{fontenc}
\usepackage[top=2cm,right=2cm,left=2cm,bottom=2cm]{geometry}

\begin{document}
\title{Теоретический минимум по оптимизации}
\author{}
\date{}

\maketitle
\thispagestyle{empty}
\vspace{-2cm}
\begin{enumerate}
\item Концепция численных методов оптимизации: скорость сходимости и критерии остановки
\item Градиентный спуск: схема метода, скорость сходимости, теоремы сходимости
\item Правила выбора шага
\item Метод Ньютона: схема метода, скорость сходимости, теоремы сходимости. Концепция квазиньютоновских методов
\item Метод сопряжённых градиентов: схема метода, скорость сходимости, теоремы сходимости. Случаи квадратичной и произвольной функции: чем похожи и чем отличаются схемы метода.
\item Задача линейного программирования: формы записи и их эквивалентность, симплекс-метод и его табличная реализация. Способы выбора начальной точки.
\item Методы проекции градиента и условного градиента: схема метода, скорость сходимости, теоремы сходимости
\item Концепция методов внутренней точки. Барьерные методы. Центральный путь и его свойства. 
\item Концепция методов штрафов и модифицированной функции Лагранжа.


\end{enumerate}
\end{document}