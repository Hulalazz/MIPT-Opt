\documentclass[12pt]{beamer}
\usepackage{../latex-sty/mypres}
\usepackage[utf8]{inputenc}
\usepackage[english]{babel}
\usepackage[T2A]{fontenc}


\expandafter\def\expandafter\insertshorttitle\expandafter{%
  \insertshorttitle\hfill%
  \insertframenumber\,/\,\inserttotalframenumber}
\title[Seminar 5]{Optimization Methods. \\
 Seminar 5. Introduction to Matrix calculus.}
\author{Alexandr Katrutsa}
\institute{Moscow Institute of Physics and Technology\\
Department of Control and Applied Mathematics} 
\date{October 3, 2016}

\begin{document}
\begin{frame}
\maketitle
\end{frame}

\begin{frame}{Reminder}
\begin{itemize}
\item Conjugate sets
\item Properties of conjugate sets
\item Farkas' lemma
\end{itemize}
\end{frame}

\begin{frame}{Basic definitions}
More details see \href{https://en.wikipedia.org/wiki/Matrix_calculus}{\color{blue}{\texttt{here}}}.
Let $f: D \rightarrow E$ be some function, then its derivative $\frac{\partial f}{\partial x} \in G$:
\begin{table}[!h]
\centering
\begin{tabular}{|c|c|c|c|}
\hline
$D$ & $E$ & $G$ & Name\\
\hline
$\bbR$ & $\bbR$ & $\bbR$ & Derivative, $f'(x)$\\
\hline
%$\bbR$ & $\bbR^n$ & $\bbR^n$ & $\frac{\partial f_i}{\partial x}$\\
%\hline
%$\bbR$ & $\bbR^{m \times n}$ & $\bbR^{m \times n}$ & $\frac{\partial f_{ij}}{\partial x}$\\
%\hline
$\bbR^n$ & $\bbR$ & $\bbR^n$ & Gradient, $\frac{\partial f}{\partial x_i}$\\
\hline
$\bbR^n$ & $\bbR^m$ & $\bbR^{n \times m}$ & Jacobian, $\frac{\partial f_i}{\partial x_j}$ \\
\hline
$\bbR^{m \times n}$ & $\bbR$ & $\bbR^{m \times n}$ & $\frac{\partial f}{\partial x_{ij}}$\\
\hline
\end{tabular}
\end{table}

Also square matrix $n \times n$ of the second derivatives $\bH = [h_{ij}]$ in case of $f: \bbR^n \rightarrow \bbR$ is called hessian and its elements equal $h_{ij} = \frac{\partial^2 f}{\partial x_i \partial x_j}$.
\end{frame}

\begin{frame}{The main technique}
\tikzset{
%Define style for boxes
punkt/.style={
rectangle,
draw=black,
minimum height=2em,
text centered}
}
\begin{tikzpicture}
\node[punkt,text width=4cm] (mat_repres) {Matrix representation of function: $f = \bx^{\T}\bA\bx$};
\node[punkt,text width=4cm,below=1cm of mat_repres] (scalar_repres) {Scalar representation of function: $f = \sum\limits_{ij}a_{ij}x_ix_j$};
\draw [->,line width=0.1em,>=stealth] (mat_repres) -- (scalar_repres);
\node[punkt,text width=5cm,right=2cm of scalar_repres] (scalar_repres_grad) {Scalar representation of gradient: $\nabla f_i = \sum\limits_{j}a_{ij}x_j$};
\draw [->,line width=0.1em,>=stealth] (scalar_repres) -- node [text width=2cm,midway,above,text centered] {$\nabla f_i = \frac{\partial f}{\partial x_i}$} (scalar_repres_grad);
\node[punkt, text width=4cm, above=1cm of scalar_repres_grad] (mat_repres_grad)  {Matrix representation of gradient: $\nabla f = \bA\bx$};
\draw [->,line width=0.1em,>=stealth] (scalar_repres_grad) -- (mat_repres_grad);
\end{tikzpicture}

\end{frame}

\begin{frame}{Examples}
\begin{enumerate}
\item Linear function: $f(\bx) = \bc^{\T}\bx$
\item Quadratic form: $f(\bx) = \frac{1}{2}\bx^{\T}\bA\bx + \mathbf{b}^{\T}\bx$
\item  $\ell_2$ norm of difference squared: $f(\bx) = \|\bA\bx - \mathbf{b}\|^2_2$
\item Determinant: $f(\bX) = \det{\bX}$
\item Trace: $f(\bX) = \Tr(\bA\bX\bfB)$
\item $f(\bx) = (\bx - \bA\bs)^{\T}\bW(\bx - \bA\bs)$
\item $f(\bA) = (\bx - \bA\bs)^{\T}\bW(\bx - \bA\bs)$
\item $f(\bs) = (\bx - \bA\bs)^{\T}\bW(\bx - \bA\bs)$
\end{enumerate}
\end{frame}

\begin{frame}{Function composition}

Let $f(\bx) = g(u(\bx))$, then $\nabla f(\bx) = \frac{\partial g}{\partial u} \frac{\partial u}{\partial \bx}$

Check dimensions and understand how to write $\frac{\partial g}{\partial u}$.

Examples:
\begin{enumerate}
\item $\ell_2$ norm of vector: $f(\bx) = \|\bx\|_2$
\item Bilinear form: $f(\bx) = u^{\T}(\bx)\bR v(\bx)$, $\bR \in \bbR^{m \times n}$
\item Exponent: $f(\bx) = - e^{-\bx^{\T}\bx}$
\end{enumerate}
\end{frame}

\begin{frame}{Recap}
\begin{itemize}
\item Derivative by scalar
\item Derivative by vector
\item Derivative by matrix
\item Derivative of function composition
\end{itemize}
\end{frame}


\end{document}