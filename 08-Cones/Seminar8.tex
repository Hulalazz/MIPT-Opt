\documentclass[12pt]{beamer}
\usepackage{../latex-sty/mypres}
\usepackage[utf8]{inputenc}
\usepackage[T2A]{fontenc}
\usepackage[russian]{babel}

\expandafter\def\expandafter\insertshorttitle\expandafter{%
  \insertshorttitle\hfill%
  \insertframenumber\,/\,\inserttotalframenumber}
\title[Семинар 8]{Методы оптимизации. \\
 Семинар 8. Разные конусы.}
\author{Александр Катруца}
\institute{Московский физико-технический институт,\\
Факультет Управления и Прикладной Математики} 
\date{24 октября 2016 г.}

\begin{document}
\begin{frame}
\maketitle
\end{frame}

\begin{frame}{Напоминание}
\begin{itemize}
\item Субдифференциал
\item Условный субдифференциал
\item Нормальный конус
\end{itemize}
\end{frame}

\begin{frame}{Конус возможных направлений}

\begin{block}{Определение}
Конусом возможных направлений для множества $G \subset \bbR^n$ в точке $\bx_0 \in G$ будем называть такое множество $\Gamma(\bx_0 | G) = \{ \bs \in \bbR^n | \bx_0 + \alpha\bs \in G, \; 0 \leq \alpha \leq \overline{\alpha}(\bs) \}$, где $\overline{\alpha}(\bs) > 0$.
\end{block}

\begin{block}{Определение для выпуклого множества}
Конусом возможных направлений для \emph{выпуклого} множества $X \subset \bbR^n$ в точке $\bx_0 \in X$ будем называть такое множество $\Gamma(\bx_0 | X) = \{ \bs \in \bbR^n | \bs = \lambda (\bx - \bx_0), \; \lambda > 0, \forall \bx \in X \}$.
\end{block}
Какая связь между нормальным конусом и конусом возможных направлений?

\end{frame}

\begin{frame}{Пример}
\begin{block}{Полезный факт}
Пусть $G = \{ \bx \in \bbR^n | \varphi_i(\bx) \leq 0, \; i = \overline{0,n-1}; \; \varphi_i(\bx) = \ba_i^{\T}\bx - b_i = 0, \; i = \overline{n, m} \}$. Тогда если $\varphi_i(\bx)$ выпуклы и множество $G$ регулярно, то \vspace{-4mm}
\[
\Gamma(\bx_0|G) = \{ \bs \in \bbR^n | \nabla \varphi_i(\bx_0)^{\T} \bs \leq 0, i \in I, \ba^{\T}_i \bs = 0, i = \overline{n,m} \}
\vspace{-4mm}
\]
и \vspace{-4mm}
\[
\Gamma^*(\bx_0|G) = \left \{ \bp \in \bbR^n \middle| \bp = \sum\limits_{i = n}^m \lambda_i\ba_i - \sum\limits_{i \in I} \mu_i \nabla\varphi_i(\bx_0) \right \},
\vspace{-4mm}
\]
где $\lambda_i \in \bbR$, $\mu_i \geq 0$, $\bx_0 \in G$ и $I = \{i: \varphi_i(\bx_0) = 0, \; i = \overline{0,n-1}\}$.
\end{block}
Найти $\Gamma(\bx_0|X)$ и $\Gamma^*(\bx_0|X)$ следующих множества:
$X = \{ \bx \in \bbR^2 | x^2_1 + 2x^2_2 \leq 3, \; x_1 + x_2 = 0 \}$.
\end{frame}

\begin{frame}{Касательный конус}
\begin{block}{Определение}
Касательным конусом к множеству $G$ в точке $\bx_0 \in \overline{G}$ называется следующее множество $T(\bx_0 |G) = \{ \lambda \bz | \lambda > 0, \; \exists \{\bx_k\} \subset G, \; \bx_k \rightarrow \bx_0, \bx_k \neq \bx_0, \; \lim\limits_{k \rightarrow \infty} \frac{\bx_k - \bx_0}{\|\bx_k - \bx_0\|_2} = \bz \}$
\end{block}

\begin{block}{Пояснение}
Касательный конус состоит из всех направлений, по которым можно сходиться к точке $\bx_0$ по последовательностям из множества $G$.
\end{block}

\begin{block}{Лемма}
Если $G$~--- выпуклое множество, то $T(\bx_0|G) = \Gamma(\bx_0|G)$.
\end{block}
\end{frame}

\begin{frame}{Полезный факт}
Пусть множество $G = \{\bx \in \bbR^n | \varphi_i(\bx) \leq 0, i = \overline{0, n-1} \; \varphi_i(\bx) = 0, i = \overline{n, m} \}$ регулярно, тогда \vspace{-4mm}
\[
T(\bx_0|G) = \{ \bz \in \bbR^n | \nabla \varphi^{\T}_i(\bx_0)\bz \leq 0, i \in I, \; \nabla \varphi^{\T}_i(\bx_0)\bz = 0, i = \overline{n,m} \}
\vspace{-4mm}
\]
и \vspace{-4mm}
\[
T^*(\bx_0|G) = \left \{ \bp \in \bbR^n \middle| \bp = \sum\limits_{i = n}^m \lambda_i \nabla \varphi_i(\bx_0) - \sum\limits_{i \in I} \mu_i \nabla \varphi_i(\bx_0) \right \},
\] 
где $\mu_i \geq 0$, $\lambda_i \in \bbR$, $I = \{i | \varphi_i(\bx_0) = 0, i = \overline{0, n-1} \}$

Пример: найти $T(\bx_0|G)$ и $T^*(\bx_0|G)$ для множества $G = \{\bx \in \bbR^2 | x_1 + x_2 \leq 1, \; x^2_1 + 2x_2^2 = 1 \}$
\end{frame}

\begin{frame}{Острый экстремум}

\begin{block}{Определение}
Точка $\bx^*$~--- точка острого минимума функции $f$ на множестве $G$, если существует $\gamma > 0$: $f(\bx) - f(\bx^*) \geq \gamma \|\bx - \bx^*\|_2, \; \forall x \in G$. 
\end{block}

\begin{block}{Лемма}
Пусть $f$~--- дифференцируемая функция на $G \subset \bbR^n$. 
Точка $\bx^*$~--- точка острого минимума функции $f$ на множестве $G$ тогда и только тогда, когда существкет такое $\alpha > 0$, что $\nabla f^{\T}(\bx^*) \bz \geq \alpha > 0, \; \bz \in T(\bx^*|G), \| \bz \|_2 = 1$.
\end{block}

\begin{block}{Примеры}
\begin{itemize}
\item $x^2_1 + x^2_2 \rightarrow \extr\limits_{G}, \quad G = \{(x_1, x_2) | x^2_1 + x_2^2 = 2, \; x_1 + x_2 \leq 1 \}$
\item $x_1 + 2x_2 \rightarrow \extr\limits_{G}$
\end{itemize}
\end{block}

\end{frame}

\begin{frame}{Резюме}
\begin{itemize}
\item Конус возможных направлений
\item Касательный конус
\item Острый экстремум
\end{itemize}
\end{frame}


\end{document}