\documentclass[12pt]{beamer}
\usepackage{../latex-sty/mypres}
\usepackage[utf8]{inputenc}
\usepackage[T2A]{fontenc}
\usepackage[russian]{babel}

\expandafter\def\expandafter\insertshorttitle\expandafter{%
  \insertshorttitle\hfill%
  \insertframenumber\,/\,\inserttotalframenumber}
\title[Семинар 9]{Методы оптимизации. \\
 Семинар 9. Сопряжённые функции}
\author{Александр Катруца}
\institute{Московский физико-технический институт,\\
Факультет Управления и Прикладной Математики} 
\date{\today}

\begin{document}
\begin{frame}
\maketitle
\end{frame}

\begin{frame}{Напоминание}
\begin{itemize}
\item Конус возможных направлений
\item Касательный конус
\item Острый экстремум
\end{itemize}
\end{frame}

\begin{frame}{Определение}
\begin{block}{Снова сопряжённое?}
\begin{itemize}
\item Ранее были рассмотрены сопряжённые (двойственные) множества и, в частности, конусы
\item Сейчас будут рассмотрены сопряжённые (двойственные) функции
\item Далее будет введена двойственная оптимизационная задача 
\end{itemize}
\end{block}

\begin{block}{Определение}
Пусть $f: \bbR^n \rightarrow \bbR$. 
Функция $f^*: \bbR^n \rightarrow \bbR$ называется сопряжённой функцией к функции $f$ и определена как
\vspace{-4mm}
\[
f^*(\by) = \sup\limits_{\bx \in dom \; f} (\by^{\T}\bx - f(\bx)).
\vspace{-3mm}
\]
Область определения $f^*$~--- это множество таких $\by$, что супремум конечен. 

\end{block}
\end{frame}

\begin{frame}{Свойства и интерпретации}
\begin{itemize}
\item Сопряжённая функция $f^*$ всегда выпукла как супремум линейных функций независимо от выпуклости $f$
\item Неравенство Юнга-Фенхеля: $\by^{\T}\bx \leq f(\bx) + f^*(\by)$
\item Если $f$~--- дифференцируема, то $f^*(\by) = \nabla f^{\T}(\bx^*)\bx^* - f(\bx^*)$, где $\bx^*$ даёт супремум.
\item Геометрический смысл
\end{itemize}
\end{frame}

\begin{frame}{Примеры}
\begin{itemize}
\item Линейная функция: $f(\bx) = \ba^{\T}\bx + b$
\item Отрицательная энтропия: $f(x) = x\log x$
\item Индикаторная функция множества $S$: $I_S(x) = 0$ iff $x \in S$
\item Норма: $f(\bx) = \|\bx\|$.
\item Квадрат нормы: $f(\bx) = \frac{1}{2}\|\bx\|^2$
\end{itemize}
\end{frame}

\begin{frame}{Резюме}

\begin{itemize}
\item Сопряжённые функции
\item Неравенство Юнга-Фенхеля и другие свойства
\item Примеры
\end{itemize}

\end{frame}

\end{document}